\documentclass[12pt, a4paper]{report}
\usepackage{amsmath, amssymb}
\usepackage{fancyhdr}
%\setlength{\headheight}{15pt}
\usepackage{geometry}
\geometry{margin=1.5in}
\usepackage[Rejne]{fncychap}
\pagestyle{fancy}
\usepackage{color}
\usepackage{blindtext}
\usepackage{import}
\usepackage{graphicx}
\usepackage{mathrsfs}
\usepackage{trfsigns}
\renewcommand{\chaptername}{Lecture}

\usepackage{import}
\usepackage{pdfpages}
\usepackage{transparent}
\usepackage{xcolor}

\newcommand{\incfig}[2][1]{%
    \def\svgwidth{#1\columnwidth}
    \import{figures}{#2.pdf_tex}
}

\pdfsuppresswarningpagegroup=1

\title{Understanding Music}
\author{Adam Carrera}
\date{January 29, 2021}
\lhead{Week \thepart}

\begin{document}
  \maketitle

  \part{Week 1}

  \chapter{Rhythm}

  Rhythm is the movement of music within time. Includes,

  \begin{itemize}
    \item Pulse (beat)
    \item Tempo
    \item Rhythmic Deviations
    \item Meter
    \item Syncopation
    \item Accents
  \end{itemize}

  \section{Pulse}

  Direct pulse refers to a beat you can hear, often the drums. Indirect pulse means the beat is felt or sensed, not directly heard.

  Examples,

  \begin{itemize}
    \item Direct pulse - Poinciana - Ahman Jamal (drums)
    \item Indirect pulse - It Never Entered My Mind - Stacey Kent (Saxophone)
  \end{itemize}

  \section{Tempo}

  Tempo is the rate of speed of a piece of music. Western music uses italian terms. Slower tempos are Largo, Lento, Adagio. Moderate tempos are andante and moderato. Faster tempos: allegro, vivace, presto. Know which of the terms generally correspond to the tempos.

  \section{Rhythmic Deviations}

  Refers to a change in the pulse or tempo. Most common are accelerando, ritardando and rubato.

  \begin{itemize}
    \item Accelerando - gradual increase in tempo
    \item Ritardando - gradual decrease in tempo
    \item Rubato - "stolen time" subtle manipulations of tempo, small increase with a small decrease
  \end{itemize}

  \section{Meter}

  Meter is the regular grouping of beats (in much Western music, meter is often thought of as the time signature). A few basic categories of meter are simple, compound, and asymmetric meters.

  \subsection{Simple Meter}

  \begin{itemize}
    \item Simple Duple meter - gorupings of 2 beats (1-2)
    \item Simple Triple meter - gorupings of 3 beats (1-2-3)
    \item Simple Quadruple meter - groupings of 4 beats (1-2-3-4)
  \end{itemize}

  \subsection{Compound Meter}

  Meters with regular groupings of sets of three beats.

  \begin{itemize}
    \item Compound Duple, two groups of three (6/8)
    \item Compound Triple, three groups of three (9/4 or 9/8)
    \item Compound Quadruple, four groups of three (12/4 or 12/8)
  \end{itemize}

  \subsection{Asymmetric Meter}

  Sometimes called additive meter, asymmetric meter includes irregular time signatures like 5/4 and 7/4, etc. They are often felt and counted as  acombination of smaller beat groupings.

  \subsection{Non-Metrical Music}

  Music that is non-metrical does not have a specific meter. One example is Gregorian Chants.

  \section{Accents}

  Accented notes or beats are emphasized in some way to stand out from the surrounding notes or beats. This can be by playing a note louder, or adding a lower or higher pitched note.

  \section{Syncopation}

  Accented notes or beats that happen in unexpected places. For example, on a "weaker" beat or on an off-beat. Offbeat is a syncopation pattern in which the accented notes/beats occur in between the main beats.

  \section{Ostinato}

  An ostinato is a short, constantly recurring melodic or rhythmic pattern. Most music with a consistent drumbeat has a rhythmic ostinato.






\end{document}
