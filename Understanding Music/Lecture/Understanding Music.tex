\documentclass[12pt, a4paper]{report}
\usepackage{amsmath, amssymb}
\usepackage{fancyhdr}
%\setlength{\headheight}{15pt}
\usepackage{geometry}
\geometry{margin=1.5in}
\usepackage[Rejne]{fncychap}
\pagestyle{fancy}
\usepackage{color}
\usepackage{blindtext}
\usepackage{import}
\usepackage{graphicx}
\usepackage{mathrsfs}
\usepackage{trfsigns}
\renewcommand{\chaptername}{Lecture}

\usepackage{import}
\usepackage{pdfpages}
\usepackage{transparent}
\usepackage{xcolor}

\setlength\parindent{0pt}

\newcommand{\incfig}[2][1]{%
    \def\svgwidth{#1\columnwidth}
    \import{figures}{#2.pdf_tex}
}

\pdfsuppresswarningpagegroup=1

\title{Understanding Music}
\author{Adam Carrera}
\date{January 29, 2021}
\lhead{Week \thepart}

\begin{document}
  \maketitle

  \part{Week 1}

  \chapter{Rhythm}

  Rhythm is the movement of music within time. Includes,

  \begin{itemize}
    \item Pulse (beat)
    \item Tempo
    \item Rhythmic Deviations
    \item Meter
    \item Syncopation
    \item Accents
  \end{itemize}

  \section{Pulse}

  Direct pulse refers to a beat you can hear, often the drums. Indirect pulse means the beat is felt or sensed, not directly heard.

  Examples,

  \begin{itemize}
    \item Direct pulse - Poinciana - Ahman Jamal (drums)
    \item Indirect pulse - It Never Entered My Mind - Stacey Kent (Saxophone)
  \end{itemize}

  \section{Tempo}

  Tempo is the rate of speed of a piece of music. Western music uses italian terms. Slower tempos are Largo, Lento, Adagio. Moderate tempos are andante and moderato. Faster tempos: allegro, vivace, presto. Know which of the terms generally correspond to the tempos.

  \section{Rhythmic Deviations}

  Refers to a change in the pulse or tempo. Most common are accelerando, ritardando and rubato.

  \begin{itemize}
    \item Accelerando - gradual increase in tempo
    \item Ritardando - gradual decrease in tempo
    \item Rubato - "stolen time" subtle manipulations of tempo, small increase with a small decrease
  \end{itemize}

  \section{Meter}

  Meter is the regular grouping of beats (in much Western music, meter is often thought of as the time signature). A few basic categories of meter are simple, compound, and asymmetric meters.

  \subsection{Simple Meter}

  \begin{itemize}
    \item Simple Duple meter - gorupings of 2 beats (1-2)
    \item Simple Triple meter - gorupings of 3 beats (1-2-3)
    \item Simple Quadruple meter - groupings of 4 beats (1-2-3-4)
  \end{itemize}

  \subsection{Compound Meter}

  Meters with regular groupings of sets of three beats.

  \begin{itemize}
    \item Compound Duple, two groups of three (6/8)
    \item Compound Triple, three groups of three (9/4 or 9/8)
    \item Compound Quadruple, four groups of three (12/4 or 12/8)
  \end{itemize}

  \subsection{Asymmetric Meter}

  Sometimes called additive meter, asymmetric meter includes irregular time signatures like 5/4 and 7/4, etc. They are often felt and counted as  acombination of smaller beat groupings.

  \subsection{Non-Metrical Music}

  Music that is non-metrical does not have a specific meter. One example is Gregorian Chants.

  \section{Accents}

  Accented notes or beats are emphasized in some way to stand out from the surrounding notes or beats. This can be by playing a note louder, or adding a lower or higher pitched note.

  \section{Syncopation}

  Accented notes or beats that happen in unexpected places. For example, on a "weaker" beat or on an off-beat. Offbeat is a syncopation pattern in which the accented notes/beats occur in between the main beats.

  \section{Ostinato}

  An ostinato is a short, constantly recurring melodic or rhythmic pattern. Most music with a consistent drumbeat has a rhythmic ostinato.







  \chapter{Harmony \& Melody}
  \section{Background}
  \begin{enumerate}
    \item Harmony = pitches heard simultaneously
    \item Melody = pitches heard in succession
  \end{enumerate}
  Harmony and melody work in different ways in different types of music.

  \section{Intervals}
  \begin{enumerate}
    \item An interval is the distance between two picthes
    \item Harmonies are based on simultaneous combinations of intervals
    \item In harmony, an interval is the distance between two simultaneously sounded pitches
  \end{enumerate}

  In western music, the \textbf{half step} is the smallest basic interval. Other music cultures recognize smaller intervals know n as quarter tones or micro-tones.

  \begin{itemize}
    \item The distance between two notes ina melody is a melodic interval
    \item The distance between two simultaneously sounding notes is a harmonic interval
  \end{itemize}

  In western music culture, intervals are classified acording to the number of half-steps they span. Intervals can be futher calssified as unison, minor, Major, diminished, augmented, perfect, and octave.

  \begin{itemize}
    \item An octave is an interval separated by 12 half steps, and is the distance between one note and the same note
    \item An octave is the only universally-recognized interval. Example, C-C.
  \end{itemize}

  \section{Dissonance and Consonance}

  \begin{itemize}
    \item Dissonant intervals are often used to create tension in Western music.
    \item A dissonant harmony is one that can sound unstable, clashing. Example: Augmented 4th
    \item Consonant intervals are often used to impart stability in Western music
    \item Stable, pleasant. Example, Octave, Perfect 5th.
  \end{itemize}

  However, different cultures have different conceptions of dissonance and consonance - what one hears as dissonant may not be thought if in that way by someone else with a different musical background.

  \section{Chords}

  \begin{itemize}
    \item Chord = any set of multiple notes heard simultaneously
    \item A triad is a three-note chord, most common basic chord type
    \item Triads are classified and named based on teh pattern of intervals they contain
    \item A Major triad consists of the first, third, and fifth notes of a Major scale played simultaneously
    \item A minor triad consists of the first, third and fifth notes of a minor scale played simultaneously
  \end{itemize}

  \section{Melody}

  A melody is a succession of different pitches one after the other. Like rhythm and harmonhy, melodies usually adhere to a particular system of music theory unique to the culture(s) from which they originate

  \section{Scales}

  \textbf{Scales} are a set of pitches in ascending and/or descending order.

  \begin{itemize}
    \item Classified by the combinations of intervals they contain
    \item Almost all cultures have scales, or ideas of scales
    \item Most used in Western, Major, minor, chromatic, pentatonic
  \end{itemize}

  \subsection{Major Scale}

  \begin{itemize}
    \item Made up of whole and half steps
    \item W-W-H-W-W-W-H
    \item C D E F G A B C
    \item Contains 7 notes (not including the top note)
    \item Stereotypically thought of as "happy"
  \end{itemize}

  \subsection{Minor Scale}

  \begin{itemize}
    \item Three different types of minor scales: natural, harmonic, and melodic
    \item All contain 7 notes, each has distinct pattern of whole and half steps
  \end{itemize}

  In music for the movies, natural minor scales are often used to highlight a sad or introspective atmosphere, and harmonic minor scales are often used to impart an "exotic" atmosphere or location. The idea of a scale or chord as being "happy" or "sad" is not an inherent universal quality of that scale or chord, but is instead a result of \textbf{cultural conditioning}. Scales and chords do not contain these qualities.

  \subsection{Chromatic Scale}

  \begin{itemize}
    \item Made up of all half-steps
    \item Contains 12 notes
    \item Contains all possible pitches within an octave
  \end{itemize}

  \subsection{Pentatonic Scale}

  \begin{itemize}
    \item A scale containing 5 notes
    \item Many forms
    \item Intervals can be larger than in Major or minor scales, often contain minor thirds
    \item Common form: C-E flat-F-G-B flat-C
  \end{itemize}

  Pentatonic scales are common in may different types of music from all over the world!

  \begin{itemize}
    \item Ancient flutes were found to have some type of pentatonic tuning
    \item Pentatonic scales are thought to be the oldest and most universal type of scale
  \end{itemize}

  Do you agree with the argument that the human brain is hardwired to hear and respond to pentatonic scales?



  \part{Week 2}

  \chapter{Intonation}

  \section{Background}

  \begin{itemize}
    \item The word intonation refers to the idea of a pitch being in or out of tune
    \item The idea of a pitch being "in tune" varies around the world
    \item Many music cultures have unique, distincy systems of intonation
  \end{itemize}

  Depending on where you are/what type of music you listen to, the pitch ideal (intonation preference) changes. Think of precise intonation versus variable intonation.

  \subsection{Precise Intonation}

  \begin{itemize}
    \item Most western music
    \item Must be "in tune"
    \item Each pitch has a specific frequency it must adhere to
    \item Music that is even slightly out of tune is "bad"
  \end{itemize}

  \subsection{Variable Intonation}

  \begin{itemize}
    \item Many cultures have a less strict idea of intonation
    \item The overall contour of a melody is valued more than landing on the exact pitches
    \item Sometimes the instruments are purposely tuned so the frequencies of similar notes are just a little bit different from each other
  \end{itemize}


\end{document}
