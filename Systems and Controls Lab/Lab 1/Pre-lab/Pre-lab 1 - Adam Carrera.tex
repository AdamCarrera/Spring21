\documentclass[12pt]{article}
\usepackage{amsmath, amssymb, amsthm}
\usepackage{import}
\usepackage{pdfpages}
\usepackage{transparent}
\usepackage{xcolor}
\usepackage{graphicx}

\usepackage{fancyhdr}

\fancyhf{}
\pagestyle{fancy}
\lhead{Carrera}
\rhead{\thepage}

\setlength\parindent{0pt}
\numberwithin{equation}{subsection}

%\usepackage{titlesec}
%\titleformat{\section}
%{\normalfont\Large\bfseries}{Exercise~\thesection}{1em}{}

\newcommand{\incfig}[2][1]{%
  \def\svgwidth{#1\columnwidth}
  \import{./figures/}{#2.pdf_tex}
}

\newcommand{\RE}{\mathrm{Re}}
\newcommand{\IM}{\mathrm{Im}}

\newcommand\ddfrac[2]{\frac{\displaystyle #1}{\displaystyle #2}}

\pdfsuppresswarningpagegroup=1

\author{Adam Carrera}
\date{February 1, 2021}
\title{MECH 4110 - Pre-lab \#1}

\begin{document}
  \maketitle

  \section{Equation of Motion Derivation}

  \subsection{Configuration 1}

  Configuration 1 consists of the pump and tank 1. Based on this, we know that the input and output flowrates are given by,

  \begin{equation}
    q_{m, \, pump} = \rho K_p v_p
  \end{equation}

  \begin{equation}
    q_m = C_d A_o \sqrt{2\rho \left( P_1 - P_2 \right)}
  \end{equation}

  where,

  \begin{align}
    P_1 &= P_a + \rho g h \\
    P_2 &= P_a
  \end{align}

  Finally, we also know that,

  \begin{equation}
    \frac{dm}{dt} = \rho A_c \dot h
  \end{equation}

  The change in mass with respect to time is given by conservation laws, we can develop the equation of motion by substituting Equations (1.1.1) - (1.1.5) and simplifying.

  \begin{equation}
    \frac{dm}{dt} = \sum q_{m, \, in} - \sum q_{m ,\, out}
  \end{equation}

  \begin{equation}
    \rho A_c \dot h = \rho K_p v_p - C_d A_o \sqrt{2\rho \left( P_a + \rho gh - P_a \right)}
  \end{equation}

  \begin{equation}
    \rho A_c \dot h = \rho K_p v_p - C_d A_o \sqrt{2\rho^2 gh}
  \end{equation}

  \begin{equation} \label{eqn:eom1}
    A_c \dot h_1 +  C_d A_o \sqrt{2gh_1} = K_p v_p
  \end{equation}

  Equation (\ref{eqn:eom1}) is a non-linear ODE relating the height to the input voltage. We have added a subscript of 1 to $ h $ so that we can differentiate between the two heights further down the line.

  \subsection{Configuration 2}

  Configuration 2 consists of both tanks, with only 1 output from the pump. The flow into Tank 2 is the flow out of Tank 1.

  \begin{equation}
    q_{m, \, in} = \rho C_{d, \, 1} A_{o, \, 1}  \sqrt{2gh_1}
  \end{equation}

  \begin{equation}
    q_{m, \, out} = \rho C_{d, \, 2} A_{o, \, 2}  \sqrt{2gh_2}
  \end{equation}

  Using the same conservation law, we can find the equation of motion for Tank 2.

  \begin{equation}
    \frac{dm}{dt} = \sum q_{m, \, in} - \sum q_{m ,\, out}
  \end{equation}

  \begin{equation}
    \rho A_{c, \, 2} \dot h_2 = \rho C_{d, \, 1} A_{o, \, 1}  \sqrt{2gh_1} - \rho C_{d, \, 2} A_{o, \, 2}  \sqrt{2gh_2}
  \end{equation}

  \begin{equation} \label{eqn:eom2}
    A_{c, \, 2} \dot h_2 + C_{d, \, 2} A_{o, \, 2}  \sqrt{2gh_2} = C_{d, \, 1} A_{o, \, 1}  \sqrt{2gh_1}
  \end{equation}





  \subsection{Configuration 3}

\end{document}
